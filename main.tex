\documentclass[11pt]{book}

\usepackage{amsmath}
\usepackage{amsfonts}
\usepackage{url}
\usepackage{amssymb,amsthm}
\usepackage[utf8]{inputenc}
%\usepackage{inconsolata}
\usepackage{sourcecodepro}

\usepackage{listings}

% Define Language
\lstdefinelanguage{futhark}
{
  % list of keywords
  morekeywords={
    do,
    else,
    fn,
    for,
    fun,
    if,
    in,
    include,
    let,
    loop,
    struct,
    then,
    type,
    val,
    while,
    with,
  },
  sensitive=true, % keywords are not case-sensitive
  morecomment=[l]{--}, % l is for line comment
  morecomment=[s]{\{-}{-\}}, % s is for start and end delimiter
%  otherkeywords={>,<,=,<=,>=,!,*,/,-,+,|,&,||,&&,==,=>},
  morestring=[b]" % defines that strings are enclosed in double quotes
}

% Define Colors
\usepackage{color}
\definecolor{eclipseBlue}{RGB}{42,0.0,255}
\definecolor{eclipseGreen}{RGB}{63,127,95}
\definecolor{eclipsePurple}{RGB}{127,0,85}

% Set Language
\lstset{
  language={futhark},
  basicstyle=\ttfamily, % Global Code Style
  captionpos=b, % Position of the Caption (t for top, b for bottom)
  extendedchars=true, % Allows 256 instead of 128 ASCII characters
  tabsize=2, % number of spaces indented when discovering a tab
  columns=fixed, % make all characters equal width
  keepspaces=true, % does not ignore spaces to fit width, convert tabs to spaces
  showstringspaces=false, % lets spaces in strings appear as real spaces
  breaklines=true, % wrap lines if they don't fit
  frame=trbl, % draw a frame at the top, right, left and bottom of the listing
  frameround=tttt, % make the frame round at all four corners
  framesep=4pt, % quarter circle size of the round corners
  numbers=left, % show line numbers at the left
  numberstyle=\small\ttfamily, % style of the line numbers
  commentstyle=\slshape\bfseries\color{eclipseGreen}, % style of comments
  keywordstyle=\bfseries\color{eclipsePurple}, % style of keywords
  stringstyle=\color{eclipseBlue}, % style of strings
  emph=[1] {
    abs,
    copy,
    concat,
    empty,
    false,
    filter,
    iota,
    map,
    partition,
    rearrange,
    reduce,
    reduceComm,
    replicate,
    reshape,
    rotate,
    shape,
    signum,
    scan,
    split,
    transpose,
    true,
    unzip,
    write,
    zip,
    zipWith,
  },
  emphstyle=[1]{\color{eclipseBlue}},
}


\usepackage{color}
\definecolor{eclipseBlue}{RGB}{42,0.0,255}
\newcommand{\soac}[1]{\texttt{\color{eclipseBlue}#1}}

\title{\bf Parallel Programming in Futhark}
\author{HIPERFIT \\ Department of Computer Science \\ University of Copenhagen (DIKU)}
\date{\today}

\begin{document}
\frontmatter
\maketitle
\chapter{Preface}

These notes ...

\tableofcontents
\mainmatter
\part{Parallel Functional Programming}
\chapter{Introduction}

\begin{enumerate}
\item Moores law, CPUs, GPUs, other parallel architectures
\item Concurrency vs parallelism
\item Task parallelism, data parallism, simd, mimd
\item Low-level languages vs high-level language approaches
\end{enumerate}

See \cite{finpar}.

\chapter{The Futhark Language}

Futhark is a pure functional data-parallel array language.  Is is both
syntactically and conceptually similar to established functional
languages, such as Haskell or Standard ML.  In contrast to these
languages, Futhark focuses less on expressivity and elaborate type
systems, but more on compilation to high-performance parallel code.
Futhark comes with language constructs for performing bulk operations
on arrays, called \textit{Second-Order Array Combinators} (SOACs),
that mirror the higher order functions found in conventional
functional languages: \texttt{map}, \texttt{reduce}, \texttt{filter},
and so forth.  In Futhark, SOACs are not merely library functions, but
built-in language features with parallel semantics, and which will
typically be compiled to parallel code.

Programming in Futhark feels similar to programming in other
functional languages.  If you know Haskell or Standard ML, you will
likely be able to read and modify most Futhark code.  For example,
this program computes the dot product $\Sigma_{i} x_{i}\cdot{}y_{i}$
of two vectors of integers:

\lstinputlisting{src/dotprod.fut}

\section{Core Language}

\section{In-Place Updates}

\begin{lstlisting}
-- A least significant digit radix sort to test out `write`.
fun radix_sort_up(xs: [n]u32) : ([n]u32,[n]i32) =
  let is = iota(n) in
  loop (p:([n]u32,[n]i32) = (xs,is)) = for i < 32 do
    radix_sort_step_up(p,i)
  in p
\end{lstlisting}


\section{Modules}

\section{When Things Go Wrong}

Futhark is a much younger and more raw language than you may be
accustomed to, and many common language features are missing.  It is
important to remember that Futhark is an \textit{on-going research
  project}, and you should not encounter the same predictability and
quality of error messages that you may be used to from more mature
languages.  In general, the limitations you will encounter will tend
to fall in three different categories:

\begin{description}
\item[Incidental] limitations are those languages features that are
  missing for no reason other than insufficient development resources.
  For example, Futhark does not support user-defined polymorphic
  functions, sum types, nontrivial type inference, or any kind of
  higher-order functions.  We know how to implement these, but simply
  have not gotten around to it yet.

\item[Essential] limitations touch upon fundamental restrictions in
  the target platform(s) for the Futhark compiler.  For example, GPUs
  do not permit dynamic memory allocation inside GPU code.  All memory
  must be pre-allocated before GPU programs are launched.  This means
  that the Futhark compiler must be able to pre-compute the size of
  all intermediate arrays (symbolically), or compilation will fail.

\item[Implementation] limitations are weaknesses in the Futhark
  compiler that could reasonably be solved.  Many implementation
  limitations, such as the inability to pre-compute some array sizes,
  or eliminate bounds checks inside parallel sections, will manifest
  themselves as essential limitations that could be worked around by a
  smarter compiler.
\end{description}

For example, consider this program:

\begin{lstlisting}
fun main(n: int): [][]int =
  map (fn i =>
         let a = iota i
         let b = iota (n-i)
         in concat a b)
  (iota n)
\end{lstlisting}

At the time of this writing, the \texttt{futhark-opencl} compiler will
fail with the not particularly illuminative error message
\texttt{Cannot allocate memory in kernel}.  The reason is that the
compiler is trying to compile the \texttt{map} to parallel code, which
involves pre-allocating memory for the \texttt{a} and \texttt{b}
array.  It is unable to do this, as the sizes of these two arrays
depend on values that are only known \textit{inside} the map, which is
too late.  There are various techniques the Futhark compiler could use
to estimate how much memory would be needed, but these have not yet
been implemented.

It is usually possible, sometimes with some pain, to come up with a
workaround.  We could rewrite the program as:

\begin{lstlisting}
fun main(n: int): [][]int =
  let big_iota = iota n
  in map (fn i =>
            let res = iota n
            let res[i:n] = big_iota[0:n-i]
            in res)
         (iota n)
\end{lstlisting}

This exploits the fact that the compiler does not generate allocations
for array slices or in-place updates.  The only allocation is of the
initial \texttt{res}, the size of which can be computed before
entering the \texttt{map}.

\chapter{Algebraic Properties of SOACs}
\begin{enumerate}
\item general reasoning principles
\item assumptions
\item fusion rules
\item list homomorphism theorem
\item let the compiler do the fusion (how to reason)
\end{enumerate}

\chapter{Parallel Cost Models}
\begin{enumerate}
\item motivation
\item memory vs compute bound
\item nested parallism and flattening
\item work and depth
\item Futhark specifics and limitations
\end{enumerate}

\part{Parallel Algorithms}

\chapter{Parallel Algorithms}
In this chapter, we will present a number of parallel algorithms for
solving a number of problems. We will make use effective use of the
SOAC parallel operators, in particular, it turns out that the
\soac{scan} operator is critical for obtaining parallel algorithms. In
fact, we shall first develop the notion of a \emph{segmented scan}
operation, which, as we shall see, can be implemented using Futhark's \soac{scan}
operator, and which in its own right is essential to many of the later
algorithms.

\section{Segmented Scan}

\lstinputlisting[firstline=7]{src/sgm_scan.fut}

\begin{enumerate}
\item segmented scan
\item radix sort
\lstinputlisting[firstline=18]{src/radix_sort.fut}
\item pseudo random numbers and sobol
\item trees
\item graphs
\item longest streak
\item segmented replication
\item histograms
\item parenthesis matching
\end{enumerate}

\chapter{Bigger Applications}
\begin{enumerate}
\item monte carlo
\item learning with stochastic gradient descent
\item stencils
\item convolutions
\end{enumerate}

\chapter{Interoperability}
\begin{enumerate}
\item python and c
\item examples: mandelbrot, life, cam, nbody
\end{enumerate}

\bibliographystyle{plain}
\bibliography{bib}

\appendix

\part{Appendices}

\chapter{Tool References}
\begin{enumerate}
\item futhark-c, futhark-opencl
\item measuring runtimes, debugging
\end{enumerate}

\end{document}

%%% Local Variables:
%%% mode: latex
%%% TeX-master: t
%%% End:
